%!TEX root = ../main.tex

\chapter{Results}\label{cha:results}

In this chapter several fingerprints will be examined in detail.
In Section \ref{sec:tft} a breakdown of the fingerprint for TitForTat is given with an explanation of its appearance.
%TOTO
Section \ref{sec:GBM} shows several plots for the strategy GoByMajority, with a varying parameter.
As the parameter changes, a continuous deformation of the plot can be observed.
Finally, in Section \ref{sec:lse} we compare the underlying data for all fingerprints of strategies within Axelrod-Python.
This data is used to produce a heat plot of strategies similarity.

It may be useful to recall that a fingerprint for a strategy is produced by playing it against varying stochastic transformations of a probe.
The parameters varied are $x$ and $y$, as seen on fingerprints in this section.
High $x$ values correspond to high cooperation.
Conversely, high $y$ values correspond to high defection.
The colour at point $\hat{x}, \hat{y}$ on the plot indicates the expected value of the strategy when played against the transformation of the probe with parameters $\hat{x}, \hat{y}$.
This expected score has been approximated as outlined in Chapter \ref{cha:implementation}.


\section{Interpretation of TFT}\label{sec:tft}
TitForTat (see Section \ref{ssec:stra_titfortat} for an explanation of its behaviour) it is one of the most well known strategies for Prisoner's Dilemma.
Also, it produces a clean and easy to understand fingerprint, shown in Figure \ref{fig:TFT}, making it an ideal place to start.

\begin{figure}[hbtp!]
\centering
\includegraphics[width = 0.6\textwidth]{../img/Numerical/Tit_For_Tat.png}
\caption{Fingerprint for TitForTat, with colour bar to add context}
\label{fig:TFT}
\end{figure}

The plot slowly transitions from a strip of blue in the top left, rotating around to a large block of red in the bottom right.
The blue area corresponds to low scores and it can be seen that this occurs for small values of $x$.
However, as $x$ increases higher scores (shown in red) quickly take over.
This is exactly as expected, TitForTat plays well in highly cooperative environments.
An immediate question is; why does the white stripe not follow the diagonal?
The main diagonal is where $x=y$ and due to the random nature of the cooperation and defections, the expected average score for TitForTat would be !!!! about half the maximum.
Due to the global minimum and maximum that TitForTat achieves, a score of !!!! is not midway, and so the white line showing the middle score is off centre.
TitForTat is able to produce better than half scores in this scenario by encouraging the underlying probe strategy to cooperate.

\section{Varying the parameter for Random}
It is possible to observe how changing a parameter for a strategy will affect its behaviour by comparing the fingerprints for each parameter.
As described in Section \ref{ssec:strat_random}, Random accepts a parameter that determines the probability of cooperation.
We can vary this parameter and create a new fingerprint each time.

\section{Alternator, Cycler(CD), Cooperator, Defector}

\section{GoByMajority for different parameters}\label{sec:GBM}

\begin{figure}[htbp!]
\subfloat[GoByMajority with memory depth 5]{\includegraphics[width = 0.5\textwidth]{../img/Numerical/Go_By_Majority_5.png}}
\subfloat[GoByMajority with memory depth 10]{\includegraphics[width = 0.5\textwidth]{../img/Numerical/Go_By_Majority_10.png}}\\
\\
\subfloat[GoByMajority with memory depth 20]{\includegraphics[width = 0.5\textwidth]{../img/Numerical/Go_By_Majority_20.png}}
\subfloat[GoByMajority with memory depth 40]{\includegraphics[width = 0.5\textwidth]{../img/Numerical/Go_By_Majority_40.png}}\\
\\
\subfloat[GoByMajority with memory depth 75]{\includegraphics[width = 0.5\textwidth]{../img/Numerical/Go_By_Majority_75.png}}
\subfloat[GoByMajority with memory depth infinite]{\includegraphics[width = 0.5\textwidth]{../img/Numerical/Go_By_Majority00.png}}
\end{figure}

\section{Random and Cycler(CDDC) with probes TFT and T42T}
The importance of probe selection will now be demonstrated.
Thus far, all fingerprints shown have used TitForTat as a probe, but this is not always enough.
For example, Random and Cycler(CDDC) produce identical fingerprints when probed with TitForTat, as shown in

\begin{figure}[htbp!]
    \centering
    \subfloat[Fingerprint for Random]{\includegraphics[width = 0.5\textwidth]{../img/Numerical/Random_TFT.png}}
    % \subfloat[Fingerprint for Cycler(CDDC)]{\includegraphics[width = 0.5\textwidth]{../img/Numerical/Cycler(CDDC)_TFT.png}}
    \caption{Fingerprints for Random and Cylcer(CDDC) when probed by TitForTat}
    \label{fig:rand_cycle_tft}
\end{figure}

However, by changing the probe to TitForTwoTats, it becomes obvious that the strategies are different.

% \begin{figure}[htbp!]
%     \centering
%     \subfloat[Fingerprint for Random]{\includegraphics[width = 0.5\textwidth]{../img/Numerical/Random_TF2T.png}}
%     \subfloat[Fingerprint for Cycler(CDDC)]{\includegraphics[width = 0.5\textwidth]{../img/Numerical/Cycler(CDDC)_TF2T.png}}
%     \caption{Fingerprints for Random and Cylcer(CDDC) when probed by TitForTwoTats}
%     \label{fig:rrand_cycle_tf2t}
% \end{figure}

\section{LSE plot and table}\label{sec:lse}
