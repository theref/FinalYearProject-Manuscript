%!TEX root = ../main.tex

\chapter{Results}\label{cha:results}

In this chapter, the implementation of a fingerprint function within the Axelrod-Python library will be examined.
This includes the addition of two strategy transformers, the Dual and JossAnn as defined in \ref{}. %TODO
Then several results will be presented where analytical fingerprints are compared with analytical ones.
A discussion that compares different fingerprints of strategies within Axelrod-Python will also be given.

\section{The Dual}
The dual of a strategy is defined such that when when the original strategy and the dual are presented with identical histories they will return opposite actions.
This relies on knowledge of how the original strategy would have behaved in a given situation, would be impractical to infer from the source code, however, the required behaviour can be achieved by having the original strategy as an attribute of the dual.
Whenever the dual has to submit a move, it can first get the original strategy to suggest what move should it would have made, and then flip that action.

\IncMargin{1.5em}
\begin{algorithm}[H]
 \KwData{A strategy}
 \KwResult{The d of the strategy}{}
  \If{First Turn}{
   create copy of original strategy\;}
  simulate original strategy\;
  update original strategy's history/internal state\;
  \Return{Flip of original strategy's move}
 \caption{The Dual of a Strategy}
\end{algorithm}\DecMargin{1.5em}

\section{Comparison of Analytical and Numerical Plots}

In figure , several analytical fingerprints from previous literature are shown. %TODO - figure ref, paper ref
Colourings and shadings are used to make certain features stand out, and an attempt to replicate this behaviour was implemented in Axelrod-Python.
The popular plotting library matplotlib has many options for different colour maps which are demonstrated in Appendix . %TODO

Using the analytical fingerprints from previous literature, and the fingerprint formulae provided alongside them, the most appropriate colour map was chosen.
The colour map Seismic was selected due to its divergent properties (although all colour maps are available within the library).
With divergent colour maps, all extreme values (high or low) are coloured, whilst mid range values are left white.
This highlights areas of interest, and in Figure it can be seen that this matches previous work very well.
