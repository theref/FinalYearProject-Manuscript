\chapter{Introduction}\label{cha:introduction}

\section{Introduction}

The Prisoner's Dilemma is a classical model in Game Theory.
It is a simple two player game where the agents must make a binary decision without communicating.
This is often presented as two suspects who have been arrested, are being interrogated separately and have the option of whether to Cooperate $(C)$ or Defect $(D)$.
Each player receives an indivual payoff that depends on the actions that have been taken.
There are 3 possible outcomes:

\begin{itemize}  
\item They both choose $C$ (mutual cooperation) and receive a payoff $R$ (Reward)
\item They both choose $D$ (mutual defection) and receive a payoff $P$ (Punish)
\item One player chooses $C$ and the other chooses $D$. The cooperator receives a payoff $S$ (Sucker) and the defector receives a payoff $T$ (Temptation).
\end{itemize}

These payoffs must satisfy the following inequalities:

\begin{equation}\label{eq:intuitive}
S < D < C < T
\end{equation}

and

\begin{equation}\label{eq:ensure_coop}
(S + T) < 2 C
\end{equation}

Equation \ref{eq:intuitive} merely fixes the payoffs in their intuitive order.
Equation \ref{eq:ensure_coop} ensures that alternating between cooperating and defecting (players take it in turns to stab each other in the back) performs no better than mutual cooperation.
These inequalities allow for many different payoff matrices to be formulated, but values of $(R, S, T, P) = (3, 0, 5, 1)$ are comonally used in literature (find references).
A more detailed explanation of the Prisoner's Dilemma is given in ~\cite{Gotts2003}.

\begin{equation}\label{eq:payoff_matrix}
% 
P = \bordermatrix{~ & C & D \cr
                  C & (3, 3) & (5, 0) \cr
                  D & (0, 5) & (1, 1) \cr}
% 
\end{equation}

It is well known that the game has one Nash Equilibrium (find a reference): both players defect.
However, by repeatadely playing the game, it appears that cooperative behaviour can emerge from the originally selfish situation.
This is called the Iterated Prisoner's Dilemma (IPD).



