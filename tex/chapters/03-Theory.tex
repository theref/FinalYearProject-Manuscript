%!TEX root = ../main.tex

\chapter{Theory}\label{cha:theory}

Many definitions will be now be presented, with the overall aim being to have a rigorous definition for a Fingerprint.
A definition of Finite State Machines will be given, and an explanation of how they relate to fingerprinting.
Then definitions for the Dual, Joss-Ann and Fingerprint as presented by Ashlock in \cite{Ashlock2004}.
It will then be shown that any strategy can be represented as a Finite State Machine, which allows the work of Ashlock to be extended to include any strategy to be used as a probe.
Finally, an example of how to construct an analytical fingerprint will be shown in Section \ref{sec:analytical-fingerprints}.
In essence this chapter presents a detailed review of the work of \cite{Ashlock2008, Ashlock2010, Ashlock2004,  Ashlock2005, Ashlock2009, Ashlock2006}.



\section{Finite State Machines (FSM)}\label{sec:fsm}

A formal definition of a Finite State Machine is given by Definition \ref{def:fsm}, but first, some motivating key characteristics of a system that can be modelled with a FSM:

\begin{itemize}
 \item The system must be describable by a finite set of states.
 \item The system must have a finite set of inputs that can trigger transitions between states.
 \item The behaviour of the system at a given point in time depends upon the current state and the input that occurs at that time.
 \item For each state the system may be in, behaviour is defined for each possible input.
 \item The system has a particular initial state.
\end{itemize}

We can make the above bullet points rigorous with the following definition:

\begin{definition}\label{def:fsm}
A \textbf{Finite State Machine} $M$ is a tuple $(S, \sigma, \delta, s_0, F)$ where
\begin{itemize}
 \item $\sigma$ is the set of symbols representing the input of $M$.
 \item $S$ is the set of states of $M$.
 \item $s_0 \in S$ is the starting state.
 \item $F \subseteq S$ is the set of final states of $M$.
 \item $\delta: S \times \sigma \rightarrow S$ is the transition function.
\end{itemize}
\end{definition}

Figure \ref{fig:Tit4TatFSM} and figure \ref{fig:PavlovFSM} show FSM representations for TitForTat and Pavlov respectively (see Sections \ref{ssec:stra_titfortat} and \ref{ssec:strat_pavlov} for an explanation of how these strategies operate).
Here nodes represent the previous action taken by the strategy and the opponent, ie. node $(D, C)$ implies that on the preceding turn, the strategy chose to Defect and the opponent chose to Cooperate.
Arcs represent the choice made by the opponent at the current turn, and lead us to the state for the next turn.

\begin{figure}[hbtp!]
\centering
\subfloat[FSM for TitForTat]{\includestandalone[width = 0.5\textwidth]{../img/Tit4TatFSM}\label{fig:Tit4TatFSM}}
\subfloat[FSM for Pavlov]{\includestandalone[width = 0.5\textwidth]{../img/PavlovFSM}\label{fig:PavlovFSM}}
\caption{Finite State Machine representations for TitForTat and Pavlov}
\end{figure}

Figures \ref{fig:Tit4TatFSM} and \ref{fig:PavlovFSM} are merely pictorial representations of Definition \ref{def:fsm}.
The set of input symbols is given by $\sigma = \{C, D\}$.
Nodes are the possible states $S$, including the initial state $s_0$.
Whilst not given explicitly, the transition function $\delta$ can be inferred from the edges of the Figures.

The next few Sections (\ref{sec:joss-ann} to \ref{sec:dual}) outline some transformations that can be applied to strategies.
In particular, Section \ref{sec:dual} relies on the strategy to be represented as a FSM as has just been defined.
Once the transformations had been presented, a formal definition of a Fingerprint can be given.



\section{The Joss-Ann}\label{sec:joss-ann}
The Joss-Ann is a basic transformation that can be applied to a strategy \cite{Ashlock2008, Ashlock2010, Ashlock2004,  Ashlock2005, Ashlock2009, Ashlock2006}.
It operates by making a probabilistic choice of Cooperation, Defection or the original move.
More formally:

\begin{definition}\label{def:joss-ann}
If $A$ is a strategy for the iterated prisoner's dilemma, then the \textbf{Joss-Anne of A} denoted by $\JA(A, x, y)$ is a transformation of that strategy.
Instead of the original behaviour, it makes move $C$ with probability $x$, move $D$ with probability $y$, and otherwise uses the response appropriate to strategy $A$ (assuming $x+y < 1$).
\end{definition}

The notation $\JA$ comes from the initials of the names Joss and Anne.
Joss was a strategy submitted to one of Axelrod’s original tournaments and it would occasionally Defect without provocation in the hopes of a slight improvement in score.
Anne is the first name of A. Stanley who suggested the addition of random Cooperation instead of random Defection \cite{Ashlock2008}.
When $x + y = 1$, the original strategy is not used, and the resulting behaviour
is a random strategy with probabilities $(x, y)$. Note that this corresponds to
Random($x$) as described in Section~\ref{ssec:strat_random}.
In more general terms, a $\JA$ strategy is an alteration of $A$ that causes the strategy to be played with random noise inserted into the responses.



\section{The Flip}\label{sec:flip}
The Flip is a simple transformation that returns the opposite of the strategy.
This is subtly different from the Dual as demonstrated in Table \ref{tab:strat-dual-flip}.

\begin{definition}\label{def:flip}
Strategy $A'$ is said to be the \textbf{Flip} of strategy $A$ if, $A$ and $A'$ return different actions for the same game history (all moves made by both players).
\end{definition}



\section{The Dual}\label{sec:dual}
The Dual is another, more complex transformation of a strategy \cite{Ashlock2008, Ashlock2010, Ashlock2004,  Ashlock2005, Ashlock2009, Ashlock2006} (and these transformations can be applied commutatively).
Traditionally, the Dual is only defined on a strategy with an FSM representation.

\begin{definition}\label{def:dual}
Strategy $A'$ is said to be the \textbf{Dual} of strategy $A$ if $A$ and $A'$ can be written as finite-state machines that are identical except that their responses are reversed.
\end{definition}

In Section \ref{sec:extend-dual} this definition will be extended to allow for the Dual to be calculated for any strategy, even ones without a natural FSM representation.
FSM representations for the Dual of TitForTat and Pavlov are given in Figures \ref{fig:Tit4TatDualFSM} and \ref{fig:PavlovDualFSM}.
The FSM representation of the Dual of Pavlov is exactly the same as the original except that the initial move has changed.
For TitForTat, the Dual plays in entirely the opposite way and all moves have been changed.

\begin{figure}[hbtp!]
\centering
\subfloat[FSM for the Dual of TitForTat]{\includestandalone[width = 0.5\textwidth]{../img/Tit4TatDualFSM}\label{fig:Tit4TatDualFSM}}
\subfloat[FSM for the Dual of Pavlov]{\includestandalone[width = 0.5\textwidth]{../img/PavlovDualFSM}\label{fig:PavlovDualFSM}}
\caption{Finite State Machine representations for the Duals of TitForTat and Pavlov}
\end{figure}

It is important to note in general the Dual is different to taking a strategy and flipping its responses.
The Dual relies on knowledge of the underlying state of the original strategy, whereas the flip does not.
This is shown in Table \ref{tab:strat-dual-flip}.

\begin{table}[htbp]
    \centering
    \begin{tabular}{l l l l}
        \toprule
        Opponent history & Pavlov & Dual & Flip \\
        \midrule
        C & C & D & D \\
        D & C & D & C \\
        D & D & C & C \\
        C & C & D & C \\
        C & C & D & D \\
        D & C & D & C \\
        C & D & C & C \\
        D & D & C & D \\
          & C & D & D \\
        \bottomrule
    \end{tabular}
    \caption{The different responses of Pavlov, Pavlov's Dual and Flipped Pavlov}
    \label{tab:strat-dual-flip}
\end{table}

The subtle difference between Dual and Flip can be highlighted further by inspecting each row individually.

\underline{Row 1} - Pavlov always plays $C$ on the first go.
Flip will change this to $D$.
Dual knows that Pavlov always plays $C$ regardless of the opponents history and so swaps to $D$.

\underline{Row 2} - In the previous round for Pavlov the strategies played $(C, C)$, and so Pavlov plays $C$ again.
For Flip, the preceding interaction was $(D, C)$, in this instance Pavlov would play $D$ again, so this gets flipped to $C$.
The previous turn for Dual was $(D, C)$ so it infers that Pavlov had $(C, C)$.
It knows that Pavlov would play $C$ and so plays $D$.

\underline{Row 3} - In the previous round for Pavlov the strategies played $(C, D)$, and so Pavlov would change to play $D$.
For Flip, the preceding interaction was $(C, D)$, in this instance Pavlov would change to $D$, so this gets flipped to play $C$ again.
The previous turn for Dual was $(D, D)$ so it infers that Pavlov had $(C, D)$.
It knows that Pavlov would play $D$ in this instance and so plays $C$.

By following this procedure we can see that the Dual responses and Pavlov's responses are always opposite.
This is achieved by the Dual inferring what the original strategy must have played and then reacting appropriately.
In the above example, Pavlov's memory is only one turn deep, so the Dual only has to infer the preceding turn, however, for more complex strategies, it may need to infer the entire history of the original strategy in order to proceed correctly.

Using the definitions presented in Sections \ref{sec:joss-ann} to \ref{sec:dual}, the Fingerprint and Double Fingerprint functions can be defined.
This provides the tools to produce a visual representation for a strategy.



\section{Fingerprint and Double Fingerprint}
The Fingerprint is a function that returns the expected score of a strategy when it plays against the Joss-Ann with varying $(x, y)$.
The double fingerprint extends this idea to $x+y \geq 1$ \cite{Ashlock2008, Ashlock2010, Ashlock2004,  Ashlock2005, Ashlock2009, Ashlock2006}.

\begin{definition}\label{def:fingerprint}
A \textbf{Fingerprint} $F_A(S, x, y)$ with $0 \leq x, y \leq 1$, $x+y \leq 1$ for strategy $S$ and \textbf{probe} $A$, is the function that returns the expected score of strategy $S$ against $\JA(A, x, y)$ for each possible $(x, y)$.
\end{definition}

\begin{definition}\label{def:double-fingerprint}
The \textbf{Double Fingerprint} $F_{AB}(S, x, y)$ with $0 \leq x, y \leq 1$ returns the expected score of strategy $S$ against $\JA(A, x, y)$ if $x+y \leq 1$, and $\JA(B, 1-y, 1-x)$ if $x+y \geq 1$.
\end{definition}

$A$ is referred to as the \textbf{lower probe} and $B$ is referred to as the \textbf{upper probe}.
This is shown in Figure \ref{fig:DualProbe}.
In \cite{Ashlock2004} it is shown that the Double Fingerprint, with an appropriate choice of lower and upper probe, is equivalent to the Fingerprint function over the unit square.
The proof in that paper will now be considered here.

\begin{theorem}\label{thm:fingerprint-unit-square}
If $A$ and $A'$ are dual strategies, then $F_{AA'}(S, x, y)$ is identical to the function $F_A(S, x, y)$ extended over the unit square.
\end{theorem}


\begin{proof}\label{prf:fingerprint-unit-square}
The Markov chain for the dual strategy $A'$ will have the same transitions as the Markov chain for the strategy $A$.
However, each entry for $x$ corresponds to the probability that the strategy $\JA(A, x, y)$ will randomly choose $C$ when it would not normally do so.
For strategy $A'$, this will occur whenever $\JA(A', x, y)$ does not randomly respond $D$, which has probability $1 - y$.

Similarly, each $y$ corresponds to the probability that the strategy $\JA(A, x, y)$ will randomly choose $D$ when it would usually respond $C$.
For strategy $A'$, this will occur whenever $\JA(A', x, y)$ does not randomly respond $C$, which has probability $1 - x$.

Thus the Markov chain for $\JA(A', x, y)$ is the Markov chain for $\JA(A, x, y)$ with the mapping $(x, y) \rightarrow (1-y, 1-x)$.
Therefore $F_{AA'}(S, x, y)$ extends to the remainder of the unit square the function given by $F_A(S, x, y)$. $\blacksquare$
\end{proof}

Theorem \ref{thm:fingerprint-unit-square} allows the Fingerprint function to naturally extend over the unit square and subsequently, the phrases Fingerprint and Double Fingerprint will be used interchangeably.
Figure \ref{fig:DualProbe} shows that in the \textcolor{sol-violet}{bottom left} region of the plot, the strategy plays against $\JA(A, x, y)$ and in the \textcolor{sol-cyan}{top right} region it plays against $\JA(A', 1-y, 1-x)$.

\begin{figure}[!hbtp]
    \begin{center}
        \includestandalone{../img/Dual}
        \caption{Where the Strategy plays against the Joss-Ann of the probe or the Joss-Ann of the Dual of the probe}\label{fig:DualProbe}
    \end{center}
\end{figure}

It is important to note that a Fingerprint is an \textbf{analytical} function for the expected score of the strategy.
If the probe cannot be represented as an FSM, the Dual of the probe cannot be constructed and the Fingerprint function cannot be computed.



\section{Analytical Fingerprints}\label{sec:analytical-fingerprints}

There are several steps to constructing the Fingerprint of a strategy and basic knowledge of Markov Chains is required.
An outline of the steps is as follows:

\begin{enumerate}
    \item Build a Markov chain model of an IPD between the strategy and probe strategy.
    \item Construct the corresponding transition matrix.
    \item Find the steady state distribution.
    \item Calculate the overall expected score by taking the dot product of the steady state distribution with the payoff vector given in Section \ref{sec:intro_pd} to obtain the fingerprint function.
    \item This can then be plotted as a heat map to make it easier to visualize.
\end{enumerate}

As an example, this process will now be applied to obtain a fingerprint for the strategy Win-Stay-Lose-Shift (sometimes referred to as Pavlov) when probed by Tit-For-Tat.

\textbf{Step 1} - Build the Markov Chain, shown in Figure \ref{fig:PavlovMC}.

\begin{figure}[ht!]
\centering
\includestandalone[width = 0.5\textwidth]{../img/PavlovMarkovChain}
\caption{Markov chain for Pavlov}
\label{fig:PavlovMC}
\end{figure}

\textbf{Step 2} - Construct the transition matrix.

\begin{equation}\label{eq:transition_matrix}
%
T = \bordermatrix{~      & (C, C) & (C, D) & (D, C) & (D, D) \cr
                  (C, C) & 1 - y  & 0      & 0      & x      \cr
                  (C, D) & y      & 0      & 0      & 1 - x  \cr
                  (D, C) & 0      & 1 - y  & x      & 0      \cr
                  (D, D) & 0      & y      & 1 - x  & 0}
%
\end{equation}

\textbf{Step 3} - Find the steady state distribution.

\begin{equation}\label{eq:steady_state}
%
\pi =
\begin{bmatrix}
\cfrac{x (1 - x)}{2y(1-x) + x(1-x) + y(1-y)}, \\
\cfrac{y (1 - x)}{2y(1-x) + x(1-x) + y(1-y)}, \\
\cfrac{y (1 - y)}{2y(1-x) + x(1-x) + y(1-y)}, \\
\cfrac{y (1 - x)}{2y(1-x) + x(1-x) + y(1-y)}
\end{bmatrix}
\end{equation}
\textbf{Step 4} - Calculate the expected score.

\begin{equation}
F = \pi \cdot
\begin{bmatrix}
3 \\
0 \\
5 \\
1
\end{bmatrix}
=
\cfrac{3x(1-x) + y(1-x) + 5y(1-y)}{2y(1-x) + x(1-x) + y(1-y)}
\end{equation}

Whilst not technically part of the Fingerprinting process, it is now useful to plot the resulting function, shown in Figure \ref{fig:pavlov_fing_ex}.

\begin{figure}[htbp!]
\centering
\includegraphics[width = 0.5\textwidth]{../img/Numerical/Win_Stay_Lose_Shift.png}
\caption{Fingerprint example for Pavlov}
\label{fig:pavlov_fing_ex}
\end{figure}

In the next section, it is shown that any strategy can be represented as a Finite State Machine.
This \textbf{original} result allows the definition of the Dual to be extended and therefore allow any strategy to be used as a probe.



\section{Proof of FSM for every strategy}\label{sec:fsm_proof}
\begin{theorem}\label{thm:fsm}
Given a deterministic strategy $\alpha$ and 2 histories $h_1, h_2$, then for all games of length $n \in \mathbb{N}$ there exists an FSM, $M$ such that $\alpha(h_1, h_2)$ can be obtained from it.
\end{theorem}


\begin{proof}\label{prf:fsm}

Let $\sigma = \{C, D\}$ be the set of symbols representing the input of $M$ and
$$ S = \bigcup_{i=0}^{n} \{C, D\}^i \times \{C, D\}^i $$
be the set of states of $M$.

Then the transition function $\delta: S \times \sigma \rightarrow S$ which maps states and actions to a new state can be defined as:
$$
\delta((h_1, h_2), a) = (h_1 \alpha(h_1, h_2), h_2a).
$$
where $a$ is the opponent's move.
Thus the 2 histories $h_1, h_2$ and an opponents action, $a$, are mapped to a new state $(h_1 \alpha(h_1, h_2), h_2a)$.
Here $h_1 \alpha(h_1, h_2)$ is the concatenation of the history, $h_1$, and the action specified by the strategy, $\alpha$.
Similarly $h_2a$ is the concatenation of the history, $h_2$, and the opponents action, $a$.

Finally the final states of $M$ denoted by $F$ is just $F = \{C, D\}^{n} \times \{C, D\}^{n} \subseteq S$.

\end{proof}

In essence, if the number of turns in a game is known, any strategy can be represented as an FSM.
Thus the FSM of a more complex strategy can be built, for example, Majority.
Majority plays in the following way:

\begin{itemize}
  \item If the opponent has Cooperated the majority of the time, Majority will Cooperate.
  \item If the opponent has Defected the majority of the time, Majority will Defect
  \item Note - the strategy shown is technically Soft Majority, if the opponents Cooperations and Defections are equal it will Cooperate. Hard Majority would Defect in this situation.
\end{itemize}

This implies that the strategy Majority requires knowledge of all previous states and therefore, could not be represented as an FSM.
However, by using Theorem \ref{thm:fsm} this is now possible if the number of turns in a game is known.
In Figure \ref{fig:MajorityFSM} the FSM for Majority in a game with 4 turns is presented.

\begin{figure}[hbtp!]
\centering
\includestandalone[width = 0.8\textwidth]{../img/MajorityFSM}
\caption{FSM for Majority in a game with 4 Turns}
\label{fig:MajorityFSM}
\end{figure}

Theorem \ref{thm:fsm} also impacts the Fingerprint process; the result allows the definition of the Dual to be extended to all strategies.
Therefore, any strategy can be used as a probe.



\section{Extended Dual}\label{sec:extend-dual}

As described in Section \ref{sec:dual}, traditionally the Dual is only defined for strategies with an FSM representation.
This implies that the Dual transformation could not be applied to a strategy without an FSM representation.
However, Theorem \ref{sec:fsm} states that every strategy can be represented as an FSM if the number of turns in the game is known.

\begin{definition}\label{def:extend-dual}
Strategy $A'$ is said to be the \textbf{Extended Dual} of strategy $A$ if, for a game of finite length $n \in N$, $A$ and $A'$ can be written as finite-state machines that are identical except that their responses are reversed.
\end{definition}

This allows the Dual transformation to be applied to all strategies and therefore, any strategy can be used as the Probe in a Fingerprint (see Definition \ref{def:fingerprint}).



\section{Conclusion}

Many definitions for FSM's and strategy transformations were presented in this Chapter, culminating in a formal definition for the Fingerprint.
An example construction of an Analytical Fingerprint was then given.
This was followed by an \textbf{original} result that showed that every strategy has an FSM representation.
This result led to the ability to extend the Dual to all strategies and therefore be able to use any strategy as a probe.
The definitions presented throughout this Chapter will now be used to develop code to Fingerprint any strategy with any probe.
This is explained in detail in Chapter \ref{cha:implementation}.

