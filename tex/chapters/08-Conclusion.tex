%!TEX root = ../main.tex

\chapter{Conclusion}\label{cha:conclusion}

As described in Chapter \ref{cha:introduction}, the Prisoner's Dilemma is a very popular model in game theory and there have been many papers written about the subject.
The game has been applied to many different research areas is often used to model systems in biology \cite{Sigmund1999}, sociology \cite{Franken2005}, psychology \cite{Ishibuchi2005}, and economics \cite{Chong2005}.
The start of this chapter will give a brief overview of the literature and particularly relevant work will be highlighted.
This is followed by an outline of how Axelrod's work is currently being reproduced by an open-source community.
Finally, an introduction to fingerprinting (a technique used for identify
similar strategies) and some necessary definitions and theorems are given at the end of the chapter.

Many definitions will be now be presented, with the overall aim being to have a rigorous definition for a fingerprint.
A definition of Finite State Machines will be given, and an explanation of how they relate to fingerprinting.
Then definitions for the Dual, Joss-Ann and Fingerprint as presented by Ashlock in \cite{Ashlock2004}.
It will then be shown that any strategy can be represented as a Finite State Machine, which allows the work of Ashlock to be extended to include any strategy to be used as a probe.
Finally, an example of how to construct an analytical fingerprint will be shown in Section \ref{sec:analytical-fingerprints}.
In essence this chapter presents a detailed review of the work of \cite{Ashlock2008, Ashlock2010, Ashlock2004,  Ashlock2005, Ashlock2009, Ashlock2006}.

In this chapter several fingerprints will be examined in detail.
In Section \ref{sec:tft} a breakdown of the fingerprint for TitForTat is given with an explanation of its appearance.
Section \ref{sec:GBM} shows several plots for the strategy GoByMajority, with a varying parameter.
As the parameter changes, a continuous deformation of the plot of the
fingerprint can be observed.
Finally, in Section \ref{sec:lse} we compare the underlying data for all fingerprints of strategies within Axelrod-Python.
This data is used to produce a heat plot of strategies similarity.
